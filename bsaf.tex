Sure! Below is a simple project structure for a React application that interacts with the Average Calculator microservice we previously discussed. This project will have a basic user interface where you can input the number ID and see the response from the microservice.

Here's the project structure:

java
Copy code
average-calculator-react/
│
├── public/
│   ├── index.html
│   └── ...
│
├── src/
│   ├── components/
│   │   ├── NumberInput.js
│   │   └── ResponseDisplay.js
│   ├── App.js
│   ├── index.js
│   └── ...
│
└── package.json
Here's what each file does:

index.html: The main HTML file where your React app is mounted.
App.js: The main component of your React app. It contains the logic for fetching data from the microservice and rendering other components.
NumberInput.js: A component for inputting the number ID.
ResponseDisplay.js: A component for displaying the response from the microservice.
You can create these files with the following content:

index.html:

html
Copy code
<!DOCTYPE html>
<html lang="en">
<head>
    <meta charset="UTF-8">
    <meta name="viewport" content="width=device-width, initial-scale=1.0">
    <title>Average Calculator</title>
</head>
<body>
    <div id="root"></div>
</body>
</html>
App.js:

javascript
Copy code
import React, { useState } from 'react';
import NumberInput from './components/NumberInput';
import ResponseDisplay from './components/ResponseDisplay';

function App() {
  const [response, setResponse] = useState(null);

  const handleResponse = (responseData) => {
    setResponse(responseData);
  };

  return (
    <div>
      <h1>Average Calculator</h1>
      <NumberInput onResponse={handleResponse} />
      <ResponseDisplay response={response} />
    </div>
  );
}

export default App;
NumberInput.js:

javascript
Copy code
import React, { useState } from 'react';
import axios from 'axios';

function NumberInput({ onResponse }) {
  const [numberId, setNumberId] = useState('');

  const handleSubmit = async (e) => {
    e.preventDefault();
    try {
      const response = await axios.get(http://localhost:9876/numbers/${numberId});
      onResponse(response.data);
    } catch (error) {
      console.error('Error fetching data:', error);
    }
  };

  return (
    <form onSubmit={handleSubmit}>
      <label>
        Number ID:
        <input type="text" value={numberId} onChange={(e) => setNumberId(e.target.value)} />
      </label>
      <button type="submit">Submit</button>
    </form>
  );
}

export default NumberInput;
ResponseDisplay.js:

javascript
Copy code
import React from 'react';

function ResponseDisplay({ response }) {
  return (
    <div>
      <h2>Response</h2>
      {response && (
        <pre>
          {JSON.stringify(response, null, 2)}
        </pre>
      )}
    </div>
  );
}

export default ResponseDisplay;
To run the project, make sure you have Node.js and npm installed. Then, navigate to your project directory in the terminal and run the following commands:

bash
Copy code
npm install axios
npm start
This will start a development server, and you can view your React app by visiting `http://localhost:3000